\documentclass[10pt,a4paper]{article}
\usepackage[utf8]{inputenc}
\usepackage{amsmath}
\usepackage{amsfonts}
\usepackage{amssymb}
\usepackage{float}
\usepackage{hyperref}
\newcommand{\Tau}{\mathcal{T}}  % \mathrm{\mathcal{T}}}

\usepackage{amsthm} % theorems
\usepackage{mathtools} % \coloneqq  ( tlmgr install mathtools )
% \usepackage[lite]{mtpro2}
\theoremstyle{definition}
\newtheorem{definition}{Definition}[section]
\newtheorem{lemma}{Lemma}[section]
\newtheorem{theorem}{Theorem}[section]

\newtheorem{exercise}{Exercise}[section]
\newcommand{\Fun}{{\mbox{Fun}}}
\newcommand{\Rel}{{\mbox{Rel}}}
\newcommand{\dom}{{\mbox{dom}}}
\newcommand{\apo}{{\mbox{'}}}  % apostrophe
\newcommand{\The}{{\mbox{The\,}}}
\newcommand{\Ob}{{\mbox{Ob}}}
\newcommand{\Mor}{{\mbox{Mor}}}
\newcommand{\Cat}{{\mbox{Cat}}}
\newcommand{\Hom}{{\mbox{Hom}}}
\newcommand{\Nat}{{\mbox{Nat}}}
\newcommand{\HomMor}{{\widetilde{\Hom}\mbox{}}}
\newcommand{\FMor}{{\widetilde{F}\mbox{}}}
\newcommand{\GMor}{{\widetilde{G}\mbox{}}}
\newcommand{\op}{{\mbox{op}}}
\newcommand{\id}{{\mbox{id}}}
\newcommand{\mor}{{\mbox{mor}}}
\newcommand{\LRA}{\Longleftrightarrow}
\newcommand{\defi}{{\mbox{def}}}
\newcommand{\eqdef}{{\stackrel{\defi}{=}}}
\newcommand{\propdef}{{\stackrel{\defi}{\ \Longleftrightarrow\ }}}
\newcommand{\inter}{{\bigcap}}
\newcommand{\interclass}{{{\bigcap}_C}}
\newcommand{\Set}{{\mbox{Set}}}
\newcommand{\myprf}{\noindent\textbf{Proof.}}
% quod erat demonstrandum :
\newcommand{\myqed}{\noindent\textbf{Q.E.D.}} 
% Logic:
\newcommand{\Spec}{{\mbox{Spec}}}
\newcommand{\Const}{{\mbox{Const}}}
\newcommand{\Var}{{\mbox{Var}}}
\newcommand{\Func}{{\mbox{Func}}}
\newcommand{\Pred}{{\mbox{Pred}}}
\newcommand{\Fm}{{\mbox{Fm}}}
\newcommand{\WF}{{\mbox{WF}}}
\newcommand{\Term}{{\mbox{Term}}}
\newcommand{\var}{{\mbox{var}}}
\newcommand{\term}{{\mbox{term}}}
\newcommand{\type}{{\mbox{type}}}
\newcommand{\ITerm}{{\mbox{ITerm}}}
\begin{document}
Our aim is to translate Isabelle's term into theorem statement, which is about derivability of certain formulas in the predicate calculus.

%Formulas are strings 
%We assume that there is one 

The following assignment of the proof constant to some term  
$$\mbox{exE}:``[| \exists x.P(x); \bigwedge x. P(x) \Longrightarrow R|] \Longrightarrow R"$$
is a notation for
$$\mbox{exE}:``\exists x.P(x) \Longrightarrow \left(\left(\bigwedge x. P(x) \Longrightarrow R\right) \Longrightarrow R\right)".$$

The following assignment
$$\mbox{impI}:``(A\Longrightarrow B)\Longrightarrow (A\longrightarrow B)"$$
gives us a hint that impI should be treated as the deduction theorem.


Let's fix some first-order signature $\sigma = \langle \Var, \Const, \Func, \Pred\rangle$ for the rest of paper.\\

$\Spec = \{\mbox{``,",``(",``)",``."}\}$
\begin{definition}
Alphabeth $\mathcal{A} = \bigcup\{\Var, \Const, \Func, \Pred, \Spec\}$.
\end{definition}


It gives rise to a set of formulas $\Fm$ in that signature.
\begin{definition}
Theory is any set of closed formulas.
\end{definition}
A requirement of formulas to be closed is a main difference between contexts and theories.
\begin{definition}
A theory $\Tau$ is consistent iff
$$ \Tau \nvdash \bot.$$
\end{definition}

\begin{definition}
A theory $\Tau$ is complete iff
$$ \forall \varphi \in \Fm. (\Tau \vdash \phi) \lor (\Tau \vdash \neg\varphi).$$
\end{definition}
% \Tau \nvdash \phi

\begin{theorem}
Let $\Tau$ be a complete first-order theory and $\alpha,\beta$ are formulas in a language of that theory, then
$$ ((\Tau \vdash \alpha)\Longrightarrow (\Tau \vdash \beta)) \Longrightarrow (\Tau \vdash (\alpha\longrightarrow \beta)),$$
where $\Longrightarrow$ is the symbol for implication of metatheory.\\
%\proof\ \\
\myprf\\
Assume H:$((\Tau \vdash \alpha)\Longrightarrow (\Tau \vdash \beta))$.\\
To prove $\Tau \vdash \alpha \longrightarrow \beta$ it is enough to show that $\Tau, \alpha \vdash \beta$ and then apply deduction theorem. So now our aim is $\Tau, \alpha \vdash \beta$.\\
The theorey $\Tau$ is complete, therefore there are two possibilities:\\
1) $\Tau \vdash \alpha$.\\
Therefore, $\Tau \vdash \beta$ by the hypothesis H and Modus Ponens.\\ Then, by weakening, $\Tau, \alpha \vdash \beta$\\
2) $\Tau \vdash \neg\alpha$.\\
Then $\Tau, \alpha \vdash \neg\alpha$ by weakening.\\ $\Tau, \alpha \vdash \bot$ by axiom for negation and MP.\\ So $\Tau, \alpha \vdash \beta$ by \textit{``ex falso quodlibet''} principle.\\
\myqed
%\qed
\end{theorem}

Theorems of Isabelle are translated into statements about some consistent and complete theory $\Tau$.

\begin{definition}%$I=J_{[]}()$ 
 A function $I$ is a translation from Isabelle/ZF's term to metatheory for first-order logic. We choose this metatheory to be also a set theory. $$I : \mbox{IsabelleZFTerms} \to \mbox{FOFormulas}.$$
 
Translation for types: $I_{\type}$\\
1) $I_{\type}(i)=\Term$\\
2) $I_{\type}(o)=\Fm$\\
3) $I_{\type}(A\Rightarrow B)=I_{type}(A)\rightarrow I_{type}(B)$ (set of functions)\\
Let's fix some function $I_\var$ from set of Isabelle's variables to .
Translation for terms: $I_{\term}$\\
1) $I_{\term}(\Phi)=\WF(\varphi) \land (\Tau \vdash \varphi)$.
Here $\varphi$ is a formula that obtained from $\Phi$. Syntactic applications in $\Phi$ are treated as real ones in $\varphi$.\\
2) $I_{\term}(X ==> Y) = (I_{\term}(X) \longrightarrow I_{\term}(Y))$ for any $X,Y\in \ITerm.$\\
3) $I_{\term}(\bigwedge S::T. Y) = \forall s \in I_\type(T). I_\term(Y)$\\
\\
% $J$ has an argument $l$ which is some ordered list of zeroes and ones. (It reperesents binary form of the numbers of variables. It will be used later to obtain a variable that is guranteed to be new in the formula.)\\
 
%$J_l(t)$ is defined inductively:\\
%$J_l(\varphi) = ``(\Tau \vdash \varphi)"$ where $\phi$ is any term of type $o$.\\
%$J_l(X ==> Y) = (J_{l+[0]}(X) \longrightarrow J_{l+[1]}(Y))$ where $X, Y$ are terms %of type $prop$ and ``+" is a lists' concatenation.\\

%Now the complicated part:\\
%We need to translate
%$J_l(\bigwedge S. P)$, where $S$ is a symbol of some type and $P$ is some term of type $prop$. (It may contain $S$ variable.)\\
%First order theories has only predicate and functional symbols in their signature.\\
%$$\mbox{Sign}(\Tau) =\mbox{Const}\cup \mbox{Func} \cup \mbox{Pred}.$$
%We assume that our object $\Tau$ theory is already extended by all possible definitions of terms and formulas. (\url{https://en.wikipedia.org/wiki/Extension_by_definitions})\\
%Example: 
%$$\bigcap :: ``(i \Rightarrow o) \Rightarrow i"$$
%For any formula $\varphi$ one can prove that 
%$$\forall x \exists ! z. (x\in z \longleftrightarrow ((\forall y. \varphi(y) \longrightarrow x\in y)\land (\exists w. \varphi(w))))$$
%(Another example is $\The :: ``(i \Rightarrow o) \Rightarrow i".$)\\
%\\

\noindent

\begin{theorem}
Let's prove the translation of
$$(\bigwedge x. P(x))==> \forall x. P(x).$$
Its translation is
$$(\forall x\in \Term. \Tau \vdash P(x))\longrightarrow \Tau \vdash \forall x. P(x).$$
\myprf\\
Since $\Tau$ is complete,
$$(\Tau \vdash \forall x. P(x))\lor (\Tau\vdash \neg \forall x. P(x)).$$
Let's assume that $(\Tau\vdash \neg \forall x. P(x)).$\\
...\\
Therefore, $(\Tau \vdash \forall x. P(x))$.\\
\myqed
\end{theorem}

...\\
...\\
...\\

\end{definition}

%\myprf
%\myqed
\end{document}