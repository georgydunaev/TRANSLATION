\documentclass[10pt,a4paper]{article}
\usepackage[utf8]{inputenc}
\input{../LATEX/gdmath}

\begin{document}
Warning: this is a draft of a document on very early stage of development. It contains a lot of inconsistencies in section "First approach".\\
\tableofcontents
\section{Common facts}

%Formulas are strings 
%We assume that there is one 

The following assignment of the proof constant to some term  
$$\exE:``[| \exists x.P(x); \bigwedge x. P(x) \Longrightarrow R|] \Longrightarrow R"$$
is a notation for
$$\textrm{exE}:``\exists x.P(x) \Longrightarrow \left(\left(\bigwedge x. P(x) \Longrightarrow R\right) \Longrightarrow R\right)".$$


The following assignment
$$\textrm{impI}:``(A\Longrightarrow B)\Longrightarrow (A\longrightarrow B)"$$
gives us a hint that impI should be treated as the deduction theorem.


Let's fix some first-order signature $\sigma = \langle \Var, \Const, \Func, \Pred\rangle$ for the rest of paper.\\

% We fix some $D\notin A$ such that 
Assume we also have some distinct symbol $D$ which invalidates every formula that contains it: $\WF(D)=\bot$.\\
We will need it later for dealing with incorrect substitution. 
$\Spec = \{\textrm{``,",``(",``)",``.",}``\land",``\lor",``\longrightarrow", ``D"\}$


\begin{definition}
Alphabeth $\mathcal{A} = \bigcup\{\Var, \Const, \Func, \Pred, \Spec\}$.
\end{definition}

This is a predicate that is true iff it is a usual first-order formula.
\begin{definition}
$\WF$ is a predicate of valence 1.\\
%*) $\WF(w) \Longrightarrow \WF(``("+w+``)")$\\
%*) $\llbrack \WF(w_1), \WF(w_2) \rrbrack \Longrightarrow \WF(``("+w_1+``\land"+w_2+``)")$\\
*) If $\WF(w)$ then $\WF(``("+w+``)")$.\\
*) If $\WF(w_1)$ and $\WF(w_2)$ then $\WF(``("+w_1+``\land"+w_2+``)")$.\\
...
\end{definition}

%\begin{definition}
%$\WF$ is a predicate of valence 1.\\
%We extend the definition of $\WF$ with additional constructor:\\
%*) $\WWF(D)=\top$
%\end{definition}

It gives rise to a set of formulas $\Fm$ in that signature.
\begin{definition}
$\Fm = \{ w\in \mathcal{A}^* : \WF(w)\}.$
\end{definition}

\begin{definition}
The substitution $t[b/x]$ of the variable with some term, is defined as usual:
\\$x[b/x] = b$
\\$y[b/x] = y$
\\$t(t_1,\dots, t_n)[b/x] = t(t_1[b/x],\dots, t_n[b/x])$
\end{definition}

\begin{definition}
Substitution of the variable with term in formula $\varphi$:
\\$\varphi[b/x]\in\mathcal{A}^*$
\\$P(t_1, \dots, t_n)[b/x]=P(t_1[b/x], \dots, t_n[b/x])$
\\$(\psi_1 \longrightarrow \psi_2)[b/x]=(\psi_1[b/x] \longrightarrow \psi_2[b/x])$
\\$(\forall x. \psi)[b/x]=(\forall x. \psi)$
\\$(\forall z. \psi)[b/x]=(\forall z. \psi[b/x])$ if $z\notin \FV(b)$;
\\$(\forall z. \psi)[b/x]=D$ otherwise.
\end{definition}


\begin{definition}
Theory is any set of closed formulas.
\end{definition}
A requirement of formulas to be closed is a main difference between contexts and theories.
\begin{definition}
A theory $\Tau$ is consistent iff
$$ \Tau \nvdash \bot.$$
\end{definition}

\begin{definition}
A theory $\Tau$ is complete iff
$$ \forall \varphi \in \Fm. (\Tau \vdash \phi) \lor (\Tau \vdash \neg\varphi).$$
\end{definition}
% \Tau \nvdash \phi

\begin{theorem}
Let $\Tau$ be a complete first-order theory and $\alpha,\beta$ are formulas in a language of that theory, then
$$ ((\Tau \vdash \alpha)\Longrightarrow (\Tau \vdash \beta)) \Longrightarrow (\Tau \vdash (\alpha\longrightarrow \beta)),$$
where $\Longrightarrow$ is the symbol for implication of metatheory.\\
%\proof\ \\
\myprf\\
Assume H:$((\Tau \vdash \alpha)\Longrightarrow (\Tau \vdash \beta))$.\\
To prove $\Tau \vdash (\alpha \longrightarrow \beta$) it is enough to show that $\Tau, \alpha \vdash \beta$ and then apply the deduction theorem. So now our aim is $\Tau, \alpha \vdash \beta$.\\
The theorey $\Tau$ is complete, therefore there are two possibilities:\\
1) $\Tau \vdash \alpha$.\\
Therefore, $\Tau \vdash \beta$ by the hypothesis H and Modus Ponens.\\ Then, by weakening, $\Tau, \alpha \vdash \beta$\\
2) $\Tau \vdash \neg\alpha$.\\
Then $\Tau, \alpha \vdash \neg\alpha$ by weakening.\\ $\Tau, \alpha \vdash \bot$ by axiom for negation and MP.\\ So $\Tau, \alpha \vdash \beta$ by \textit{``ex falso quodlibet''} principle.\\
\myqed
%\qed
\end{theorem}

% Therefore, $[b/x] \in \mathcal{A}^*$.

\section{First approach(wrong)}
Our aim is to translate Isabelle's term into theorem statement, which is about derivability of certain formulas in the predicate calculus.


\begin{definition}
Substitution is defined on Term and Fm. We will expand it on functions. %

Assume $f \in (A \to B)$, $t\in\Term$, $v\in\Var$.
$$f[t/v] = \{\langle a,b[t/v]\rangle: \langle a,b\rangle \in f\}$$
\end{definition}

\begin{definition}
$$(X \rsto Y) = \{f:X \to Y |  \forall t\in\Term.\forall v\in\Var.\forall x\in X. f(x)[t/v] = f(x[t/v])\}$$
\end{definition}

Theorems of Isabelle are translated into statements about some consistent and complete theory $\Tau$.

\begin{definition}
We will call the following symbol a ``weak derivability".
$$(\Tau \vdash_w \psi) \propdef (\WF(\psi) \longrightarrow (\Tau \vdash \psi))$$
\end{definition}


\begin{definition}%$I=J_{[]}()$ 
 A function $I$ is a translation from Isabelle/ZF's term to metatheory for first-order logic. We choose this metatheory to be also a set theory. $$I : \textrm{IsabelleZFTerms} \to \textrm{FOFormulas}.$$
 
Translation for types: $I_{\type}$
\\1) $I_{\type}(i)=\Term$
\\2) $I_{\type}(o)=\Fm$
\\3) $I_{\type}(A\Rightarrow B)=I_{type}(A)\rightarrow I_{type}(B)$ (set of functions)
\\Let's fix some function $I_\var$ from the set of Isabelle's variables to the set $\Var$.
\\Translation for terms: $I_{\term}$
\\1) $I_{\term}(\Phi)= (\Tau \vdash_w \varphi)$
% (\WF(\varphi) \longrightarrow (\Tau \vdash \varphi))$.
\\Here $\varphi$ is a formula that obtained from $\Phi$.
\\Syntactic applications in $\Phi$ are treated as real ones in $\varphi$.
\\2) $I_{\term}(X ==> Y) = (I_{\term}(X) \longrightarrow I_{\term}(Y))$ for any $X,Y\in \ITerm.$\\
3) $I_{\term}(\bigwedge S::T. Y) = \forall s \in I_\type(T). I_\term(Y)$\\
\end{definition}

% $J$ has an argument $l$ which is some ordered list of zeroes and ones. (It reperesents binary form of the numbers of variables. It will be used later to obtain a variable that is guranteed to be new in the formula.)\\
 
%$J_l(t)$ is defined inductively:\\
%$J_l(\varphi) = ``(\Tau \vdash \varphi)"$ where $\phi$ is any term of type $o$.\\
%$J_l(X ==> Y) = (J_{l+[0]}(X) \longrightarrow J_{l+[1]}(Y))$ where $X, Y$ are terms %of type $prop$ and ``+" is a lists' concatenation.\\

%Now the complicated part:\\
%We need to translate
%$J_l(\bigwedge S. P)$, where $S$ is a symbol of some type and $P$ is some term of type $prop$. (It may contain $S$ variable.)\\
%First order theories has only predicate and functional symbols in their signature.\\
%$$\textrm{Sign}(\Tau) =\textrm{Const}\cup \textrm{Func} \cup \textrm{Pred}.$$
%We assume that our object $\Tau$ theory is already extended by all possible definitions of terms and formulas. (\url{https://en.wikipedia.org/wiki/Extension_by_definitions})\\
%Example: 
%$$\bigcap :: ``(i \Rightarrow o) \Rightarrow i"$$
%For any formula $\varphi$ one can prove that 
%$$\forall x \exists ! z. (x\in z \longleftrightarrow ((\forall y. \varphi(y) \longrightarrow x\in y)\land (\exists w. \varphi(w))))$$
%(Another example is $\The :: ``(i \Rightarrow o) \Rightarrow i".$)\\
%\\

\noindent
\pagebreak
\begin{theorem} allI(A)
(HERE $\vdash_w=\vdash$) 
Remember that $P(x)$ is just an apllication of the function P. It's not like the traditional notation $\varphi(x_1,\dots,x_n)$ that means that $\{x_1,\dots,x_n\}\subseteq \FV(\varphi)$.
\\Let's prove the translation of
$$\allI:``\left(\bigwedge t. P(t)\right)==> \forall x. P(x)".$$
Its translation is
$$\forall P:\Term\to\Fm. (\forall t\in \Term. \Tau \vdash P(t))\longrightarrow \Tau \vdash \forall x. P(x).$$
\myprf
\\We have a hypothesis H:$(\forall t\in \Term. \Tau \vdash P(t))$.
\\Since $\Tau$ is complete,
$$(\Tau \vdash \forall x. P(x))\lor (\Tau\vdash \neg \forall x. P(x)).$$
Let's assume that $(\Tau\vdash \neg \forall x. P(x)).$
\\It means $(\Tau\vdash \exists x. \neg P(x)).$
\\$\Tau$ satisfies Henkin axioms. (e.g. $\Tau$ is a theory in a language that is a witnessing expansion of some theory) 
\\$\Tau$ is a Henkin theory, so there exists a constant $c\coloneqq c_{\neg P(x)}$ such that $\Tau\vdash (\exists x. \neg P(x))\longrightarrow \neg P(c_{\neg P(x)})$. Therefore, $\Tau\vdash \neg P(c)$ by MP.
\\Moreover, $c\in \Const \subseteq \Term$. $\stackrel{H}{\Longrightarrow} \Tau \vdash P(c)$.
\\Theory proves both statement P(c) and it's negation, so $\Tau\vdash \bot$. This contradicts the assumption that $\Tau$ is consistent.
\\Therefore, the other possibility holds: $(\Tau \vdash \forall x. P(x))$.
\\\myqed
\end{theorem}

\begin{theorem} allI(B)
(ERRONEOUS VARIANT)\\
Let's prove the translation of
$$\allI:``\left(\bigwedge t. P(t)\right)==> \forall x. P(x)".$$
Its translation is
$$(\forall t\in \Term. \Tau \vdash_w \varphi[t/x])\longrightarrow \Tau \vdash_w (\forall x. \varphi).$$
\myprf
\\We have a hypothesis H:$(\forall x\in \Term. \WF(\varphi) \longrightarrow \Tau \vdash_w \varphi)$.
\\We also have hypothesis H2:$\WF(\forall x. \varphi)$.
\\Since $\Tau$ is complete and $\WF(\forall x. \varphi)$,
$$(\Tau \vdash \forall x. \varphi)\lor (\Tau\vdash \neg \forall x. \varphi).$$
Let's assume that $(\Tau\vdash \neg \forall x. \varphi).$
\\It means $(\Tau\vdash \exists x. \neg \varphi)$ by De Morgan's rule.
\\$\Tau$ is a Henken theory, so there exists a constant $c$ such that $\Tau\vdash \neg \varphi[c/x]$.
\\We now that $\WF(\varphi[c/x])$ since $\WF(\varphi)$ and the fact that the substitution of variable with constant is always correct.
\\Moreover, $c\in \Const \subseteq \Term$. $\stackrel{H}{\Longrightarrow} \Tau \vdash \varphi[c/x]$.
\\Theory proves both statement $\varphi[c/x]$ and it's negation, so $\Tau\vdash \bot$. This contradicts the assumption that $\Tau$ is consistent.
\\Therefore, the other possibility holds: $(\Tau \vdash \forall x. P(x))$.
\\\myqed
\end{theorem}

\begin{theorem} spec(A)
(GOOD VARIANT)\\
Consider IFOL's proof constant
$$\spec: ``\forall x. P(x) ==> P(t)".$$
%$$(\bigwedge t. P(t))==> \forall x. P(x).$$
The translation of its term is
$$\forall P:\Term\rsto\Fm.(\Tau \vdash_w \forall x. P(x))\longrightarrow  (\forall t\in \Term. \Tau \vdash_w P(t)).$$
Let's prove it.\\
\myprf\\
We have a hypothesis H:$\Tau \vdash (\forall x. P(x))$.
%\\1)Let's show that $\WF(\forall x. \varphi [x/x])$.
%\\It is from $x\in \Var$ and $\WF(\varphi [x/x])$.
%\\The last is true since $\WF(\varphi)$ and the fact that substitution $[ x / x ]$ is always correct.
%\\2)
\\We fixed some $t\in \Term$.
\\We also have $\WF(P(t))$
\\by the ..., $t=x[t/x]$.
\\Obviously, $P(t)=P(x[t/x])$
\\By the property of P, $P(x[t/x])=P(x)[t/x]$.
\\So $\WF(P(x)[t/x])$.
\\Therefore $P(x)$ is free for substitution of $x$ with $t$.
\\Our aim is $\Tau \vdash P(t)$.
\\There is a fact that if $\varphi$ is free for substitution of $x$ with $b$, then 
$\Tau \vdash (\forall x.\varphi)\longrightarrow \varphi[b/x]$.
\\$\varphi\coloneqq P(x)$
% \\BAD We need to show that $\phi$ is free for substitution of $x$ with $t$.
% \\BAD We already have it from $\WF(\varphi[t/x])$.
%\\$P: \Term \to \Fm$. (Not very good representation...)\\
%We need to rename every bound variable in $P(x)$, such that term $t$ will not contain any bound variable.
\\So, by modus ponens with H and the previous expression we have $\Tau \vdash \varphi[t/x]$.
It means that $\Tau \vdash P(t)$.
\\\myqed
\end{theorem}

\begin{theorem} spec(B)
(ERRONEOUS VARIANT)\\
Assume that $I(P)=\varphi$.
Consider the proof constant
$$\spec: ``\forall x. P(x) ==> P(t)".$$
%$$(\bigwedge t. P(t))==> \forall x. P(x).$$
The translation of its term is
%$$(\Tau \vdash \forall x. P(x))\longrightarrow  (\forall t\in \Term. \Tau \vdash P(t)).$$
$$(\Tau \vdash_w \forall x. \varphi[x/x])\longrightarrow  (\forall t\in \Term.  \Tau \vdash_w \varphi[t/x]).$$
$$(\forall t\in \Term.(\WF(\forall x. \varphi[x/x]) \longrightarrow \Tau \vdash \forall x. \varphi[x/x])\longrightarrow   (\WF(\varphi[t/x]) \longrightarrow \Tau \vdash \varphi[t/x])).$$
Let's prove it.\\
\myprf\\
We have a hypothesis H:$\Tau \vdash (\forall x. \varphi(x))$.
\\1)Let's show that $\WF(\forall x. \varphi [x/x])$.
\\It is from $x\in \Var$ and $\WF(\varphi [x/x])$.
\\The last is true since $\WF(\varphi)$ and the fact that substitution $[ x / x ]$ is always correct.
\\2)We fixed some $t\in \Term$.
\\We also have $\WF(\varphi[t/x])$
\\Our aim is $\Tau \vdash \phi[t/x]$.
\\If $\varphi$ is free for substitution of $x$ with $b$, then 
$\Tau \vdash (\forall x.\varphi)\longrightarrow \varphi[b/x]$.
\\We need to show that $\phi$ is free for substitution of $x$ with $t$.
\\We already have it from $\WF(\varphi[t/x])$.
%\\$P: \Term \to \Fm$. (Not very good representation...)\\
%We need to rename every bound variable in $P(x)$, such that term $t$ will not contain any bound variable.
\\So, by modus ponens and previous expression we have $\Tau \vdash \varphi[t/x]$.
\\\myqed
\end{theorem}

\begin{lemma}
$\psi[t/v]=\psi[x/v][t/x]$ if $``x"\notin\FV(\psi)$ and substitutions are correct.
\\\myprf
\\\dots
\\\myqed
\end{lemma}
\begin{theorem} spec(C)\\
Consider IFOL's proof constant
$$\spec: ``\forall x. P(x) ==> P(t)".$$
%$$(\bigwedge t. P(t))==> \forall x. P(x).$$
\\During the proof we will use following assumptions:
\\Assumption0: $t\in\Term$
\\Assumption1: There are some formula $\psi$ and some variable $v$, such that $P$ is a partial function $(\lambda w\in\Term. \psi[w/v])$.
% \\(TODO: define this function set-theoretically)
\\(side aim: we may prove that P respects some substitutions)
\\Assumption2: we need to be assured that every application has a value. Recursively we obtain the set of applications of atomic Isabelle's symbols: $\{``P(x)", ``P(t)"\}$. Therefore, our assumption is
\\$\{``x", t\}\subseteq\dom(P)$.
\\Assumption3: free variables of arguments and formula are not coinside.
\\a)$\FV(``x")\cap\FV(\psi)=\varnothing$ (i.e. $``x"\notin\FV(\psi)$),
\\b)$\FV(t)\cap\FV(\psi)=\varnothing$.
% $``x"\in\FV(P)$ and $``t"\in\FV(P)$.
\\So the possible translation of spec's term is
% \forall P:\Term\rsto\Fm.
%$$\forall P\in\{\dots\}.$$
$$\forall \psi \in \Fm.$$
$$P=(\lambda w\in\Term. \psi[w/v])\longrightarrow$$
$$\FV(``x")\cap\FV(\psi)=\varnothing \longrightarrow$$
$$\forall t\in \dom(P). $$
$$\FV(t)\cap\FV(\psi)=\varnothing  \longrightarrow$$
$$(\Tau \vdash \forall x. P(x))\longrightarrow  \Tau \vdash P(t).$$
Let's prove it.
\\\myprf
\\We have a hypothesis H:$\Tau \vdash (\forall x. P(x))$.
%\\1)Let's show that $\WF(\forall x. \varphi [x/x])$.
%\\It is from $x\in \Var$ and $\WF(\varphi [x/x])$.
%\\The last is true since $\WF(\varphi)$ and the fact that substitution $[ x / x ]$ is always correct.
%\\2)
\\We fixed some $t\in \dom(P)\subseteq\Term$.
\\Therefore $P(t)\in\Fm$ and $\WF(P(t))$.(!)
\\(by the way, $t\notin\dom(P)\longrightarrow P(t)=\varnothing$, but $\varnothing\notin\Fm$, since it is a word of zero length.)
\\There is a fact that if $\varphi$ is free for substitution of $x$ with $b$, then 
$\Tau \vdash (\forall x.\varphi)\longrightarrow \varphi[b/x]$.
\\In order to use it, we need to prove that $P(t)=P(x)[t/x]$
\\We need to show that $\psi[t/v]=\psi[x/v][t/x]$(*) by lemma.
\\It is true, because: % $``x"\notin\FV(\psi)$ and substitutions are correct.
\\a) Substitutions $[t/v]$ and $[x/v]$ are correct by other assumptions;
%\\But what if $\psi$ contains $x$?!
\\b)$``x"\notin\FV(\psi)$ by assumption 3(b).
%\\-----------------------------
%\\By the definition of substitution in terms, $t=x[t/x]$.
%\\Obviously, $P(t)=P(x[t/x])$
%\\By the property of P, $P(x[t/x])=P(x)[t/x]$.
%\\So $\WF(P(x)[t/x])$.
\\Therefore $P(x)$ is free for substitution of $x$ with $t$.
\\Our aim is $\Tau \vdash P(t)$.
% \\BAD We need to show that $\phi$ is free for substitution of $x$ with $t$.
% \\BAD We already have it from $\WF(\varphi[t/x])$.
%\\$P: \Term \to \Fm$. (Not very good representation...)\\
%We need to rename every bound variable in $P(x)$, such that term $t$ will not contain any bound variable.
\\$\varphi\coloneqq P(x)$ (therefore $\varphi=\psi[x/v]$)
\\$\Tau \vdash (\forall x.\psi[x/v])\longrightarrow \psi[x/v][b/x]$.
\\So, by modus ponens with H and the previous expression we have $\Tau \vdash \psi[x/v][t/x]$.
It means that $\Tau \vdash P(t)$, by (*).
\\\myqed
\end{theorem}


% 1) every variable and constant in Isabelle's term should have some 
% fixed type? without variable
% formula shoul
$$\exI:$$

\begin{theorem} exE(A)\\
Let's prove the translation of
$$\exE:``[| \exists x.P(x); \bigwedge x. P(x) \Longrightarrow R|] \Longrightarrow R"$$
Its translation is
$$\forall P:\Term\to\Fm.\forall R:\Fm. \Tau\vdash\exists x.P(x) \longrightarrow ((\forall x\in\Term. \Tau\vdash P(x) \longrightarrow \Tau\vdash R)\longrightarrow \Tau\vdash R)$$
\end{theorem}
\pagebreak
\section{Second approach}


\begin{definition}%$I=J_{[]}()$ 
 A function $I$ is a translation from Isabelle/ZF's term to metatheory for first-order logic. We choose this metatheory to be also a set theory. $$I : \textrm{IsabelleZFStatements} \to \textrm{FOFormulas}.$$
 
Evaluation: 
$$v: \textrm{ITVar} \to \Set$$
 
Translation for statemens: $I_{\statement}$
$$I_\statement(a:A) = (I_{\term} \in I_{\type})$$
 
Translation for types: $I_\type$
\begin{enumerate}
\item $I_{\type}('a) = v('a)$
\item $I_{\type}(\prop)=\{0,1\}$
%\item $I_{\type}(A=>B)=I_{\type}(A) $
\item $I_{\type}(i)=\Term$
\item $I_{\type}(o)=\Fm$
\item $I_{\type}(A\Rightarrow B)=I_{\type}(A)\rightarrow I_{\type}(B)$ (set of functions)
\end{enumerate}

Translation for terms: $I_\term$\\
Translation for statements: $I_\statement$

\begin{enumerate}
\item $I_\statement(==>::\prop=>\prop=>\prop)\eqdef$\\$=(x\in 2\mapsto(y\in 2\mapsto 
(1-x)\cup y))\in (2\to(2\to2))$
\item $I_\statement(!!::('a=>\prop)=>\prop)=$\\$=(g\in(v('a)\to2)\mapsto \{w\in1: \forall q\in2.g(q)=1\})\in ((v('a)\to2)\to2)$

\item $I_\statement(\Truej::o=>\prop)=$\\
$=I_{\term}(\Truej)\in I_\type(o=>\prop)$\\
$=1_{\{\varphi\in\Fm : \Tau \vdash \varphi\}} \in(\Fm\to 2)$
\item $I_\statement(\False::o)=\bot\in\Fm$
\item $I_\statement(\longrightarrow::o=>o=>o)=$\\$=(\psi_1\in\Fm\mapsto(\psi_2\in\Fm\mapsto ``("+\psi_1+``\longrightarrow"+\psi_2+``)"))\in(\Fm\to(\Fm\to\Fm))$
\item 

For the next part we shall somehow choose variable and formula should depend on term with some restirctions:\\
1)it may be computable in sense that there is a simply typed lambda term that represents it etc.. But it sounds too syntactical.
2)It may be something like this $\epsilon (v\in\Var\setminus\FV(?))$, but it's too complicated../

$I_\statement(\forall::((i=>o)=>o))=$\\$=(g\in(\Term\to\Fm)\mapsto ``(\forall"+?+``."+*+``)")\in((\Term\to\Fm)\to\Fm)$
\item ...

%(\varphi \mapsto \Tau \vdash \varphi) $
%\item $I_{\term}(A=>B)=I_{\type}(A) $
%\item $I_{\term}(i)=\Term$
%\item $I_{\term}(o)=\Fm$
%\item $I_{\term}(A\Rightarrow B)=I_{type}(A)\rightarrow I_{type}(B)$ (set of functions)
\end{enumerate}

Translation for proof: $I_\iproof$\\
Translation for judgements: $I_\judgement$\\
\begin{enumerate}
\item $I_\judgement(\spec: ``\forall x. P(x) ==> P(t)")=$\\
$I_\iproof(\spec)\in I_\term(\forall x. P(x) ==> P(t)")=$\\
\end{enumerate}

\begin{theorem}
$$\spec: ``\forall x. P(x) ==> P(t)"$$
\end{theorem}

%1) $I_{\type}(i)=\Term$
%\\2) $I_{\type}(o)=\Fm$
%\\3) $I_{\type}(A\Rightarrow B)=I_{type}(A)\rightarrow I_{type}(B)$ (set of functions)
%-------------------------------------

%Let's fix some function $I_\var$ from the set of Isabelle's variables to the set $\Var$.
%\\

%Translation for terms: $I_{\term}$
%\\1) $I_{\term}(\Phi)= (\Tau \vdash_w \varphi)$
% (\WF(\varphi) \longrightarrow (\Tau \vdash \varphi))$.
%\\Here $\varphi$ is a formula that obtained from $\Phi$.
%\\Syntactic applications in $\Phi$ are treated as real ones in $\varphi$.
%\\2) $I_{\term}(X ==> Y) = (I_{\term}(X) \longrightarrow I_{\term}(Y))$ for any $X,Y\in \ITerm.$\\
%3) $I_{\term}(\bigwedge S::T. Y) = \forall s \in I_\type(T). I_\term(Y)$\\
\end{definition}
 
\section{Third approach}

First of all, let's fix some set $\Un$. (This will be our universe of discourse.)
\begin{definition}%$I=J_{[]}()$ 
 A function $I$ is a translation from Isabelle/ZF's types, terms and proofs.
 
% to metatheory for first-order logic. We choose this metatheory to be also a set theory. $$I : \textrm{IsabelleZFStatements} \to \textrm{FOFormulas}.$$

\noindent
Evaluation: 
$$v: \textrm{ITVar} \to \Set$$
 
\noindent
Translation for statemens: $I_{\statement}$
$$I_\statement(E::Y) = (I_{\term}(E) \in I_{\type}(Y))$$
 
Translation for types: $I_\type$
\begin{enumerate}
\item $I_{\type}('a) = v('a)$
\item $I_{\type}(\prop)=\{0,1\}$
%\item $I_{\type}(A=>B)=I_{\type}(A) $
\item $I_{\type}(i)=\Un$
\item $I_{\type}(o)=\{0,1\}=2$
\item $I_{\type}(A\Rightarrow B)=I_{\type}(A)\rightarrow I_{\type}(B)$ (set of functions)
\end{enumerate}

\noindent
Translation for terms: $I_\term$\\
Translation for statements: $I_\statement$

\begin{enumerate}
\item $I_\statement(==>::\prop=>\prop=>\prop)\eqdef$\\$=(x\in 2\mapsto(y\in 2\mapsto 
(1-x)\cup y))\in (2\to(2\to2))$
\item $I_\statement(!!::('a=>\prop)=>\prop)=$\\$=(g\in(v('a)\to2)\mapsto \{w\in1: \forall q\in2.g(q)=1\})\in ((v('a)\to2)\to2)$ %ok
\item $I_\statement(\Truej::o=>\prop)=$
\\$=\id_2\in(2\to2)$
%\\$=I_{\term}(\Truej)\in I_\type(o=>\prop)$
%\\$=1_{\{\varphi\in\Fm : \Tau \vdash \varphi\}} \in(\Fm\to 2)$
\item $I_\statement(\False::o)=0\in2$
\item $I_\statement(\longrightarrow::o=>o=>o)=$
\\$=(x\in 2\mapsto(y\in 2\mapsto 
(1-x)\cup y))\in (2\to(2\to2))=$
\\$=(x\in 2\mapsto(y\in 2\mapsto \{m\in1: x=0 \lor y=1\}))$
%\\$=(\psi_1\in\Fm\mapsto(\psi_2\in\Fm\mapsto ``("+\psi_1+``\longrightarrow"+\psi_2+``)"))\in(\Fm\to(\Fm\to\Fm))$
\item $I_\statement(\forall::((i=>o)=>o))=$
%\\$=(g\in(\Term\to\Fm)\mapsto ``(\forall"+?+``."+*+``)")\in((\Term\to\Fm)\to\Fm)$
\\$=(g\in(\Un\to2)\mapsto \{w\in1: \forall q\in\Un.g(q)=1\})\in ((\Un\to2)\to2)$
\item ...

%(\varphi \mapsto \Tau \vdash \varphi) $
%\item $I_{\term}(A=>B)=I_{\type}(A) $
%\item $I_{\term}(i)=\Term$
%\item $I_{\term}(o)=\Fm$
%\item $I_{\term}(A\Rightarrow B)=I_{type}(A)\rightarrow I_{type}(B)$ (set of functions)
\end{enumerate}

\noindent
Translation for proof: $I_\iproof$
$$I_\iproof(p)=\varnothing$$
Translation for judgements: $I_\judgement$
$$ I_\judgement(p:E) = (I_\iproof(p) \in I_\term(E)) $$

\end{definition}

\begin{enumerate}
\item $(P:i=>o)$, $(t:i)$\\
$\psi:\Un\to2$, $x \in \Un$\\
\\$I_\judgement(\spec: ``\forall x. P(x) ==> P(t)")=$
\\$=I_\iproof(\spec)\in I_\term(\forall x. P(x) ==> P(t))=$
\\$=I_\iproof(\spec)\in (1-I_\term(\forall x. P(x)))\cup I_\term(P(t))$
\\$=I_\iproof(\spec)\in (1-
\{w\in1: \forall q\in\Un. I_\term(P)(q)=1\}
)\cup I_\term(P(t))$
\\$=\varnothing\in ((1-
\{w\in1: \forall q\in\Un. \psi(q)=1\}
)\cup \psi(x))$
\\

$p:\Un\to2$, $x \in \Un$
\\$I_\term(\forall x. P(x) ==> P(t))=$
\\$=\{\_\in1:I_\term(\forall x. P(x))=0 \lor I_\term(P(t))=1\}=$
\\$=\{\_\in1:(I_\term(\forall x. P(x))=0) \lor (p(x)=1)\}=$
\\$=\{\_\in1:(\{\_\in1: \forall q\in\Un. p(q)=1\}=0) \lor (p(x)=1)\}=$
\\$=\{\_\in1:(\neg\forall q\in\Un. p(q)=1) \lor (p(x)=1)\}$
\\

\noindent $(I_\term(\forall x. P(x) ==> P(t))=1) \Longleftrightarrow $
\\$\Longleftrightarrow \forall p:\Un\to2.\forall s \in \Un. (\forall q\in\Un. p(q)=1) \longrightarrow (p(s)=1)$
\\For example,
$$p(s)=\left\{
\begin{matrix}
1 &\textrm{if }\Tau \vdash \varphi(c_s/v),\\
0 &\textrm{otherwise}
\end{matrix}
\right..$$
Here $c_s$ is some constant. It's intended interpretation is $I(c_s)=s$. It means that we have to use a proper class of constants, that sounds bad. Maybe use the set of set which are somehow represented by terms?..
%What can we use 
%What if instead of some Grothendieck universe $\Un$ we choose a constructible universe $L$? Or $\{u\in\Un : \exists t\in\Term. I(t)=y\}$?
\\

\noindent Some of the proof constants become useless, for example allI:
\\$(I_\term((!!t. P(t))==>(\forall x. P(x)))=1) \Longleftrightarrow $
\\$\Longleftrightarrow \forall p:\Un\to2.\forall s \in \Un. (\forall t\in\Un. p(t)=1)\longrightarrow (\forall x\in\Un. p(x)=1)$ 
\end{enumerate}
%I_\term(\forall x. P(x))

\begin{theorem}
$$\spec: ``\forall x. P(x) ==> P(t)"$$
it's translation is
$$
\psi:\Un\to2\ \&\ x \in \Un \Longrightarrow
\varnothing\in ((1-
\{w\in1: \forall q\in\Un. \psi(q)=1\}
)\cup \psi(x))$$
\myprf
\\$\psi(x)\in 2$. Therefore $\psi(x)=0$ or $\psi(x)=1$.
\\1) $\psi(x)=1$. $\Rightarrow$
\\$\Rightarrow$ $\varnothing\in\psi(x)$ (by def. of 1). $\Rightarrow$
\\$\Rightarrow$ $\varnothing\in Q \cup \psi(x)$ (by property of $\cup$),
\\where $Q=(1-\{w\in1: \forall q\in\Un. \psi(q)=1\})$.
\\2) $\psi(x)=0$. $\Rightarrow$
\\$\Rightarrow$ $\exists q\in\Un. \psi(q)\neq 1$ $\Rightarrow$
\\$\Rightarrow$ $\{w\in1. \forall q\in\Un. \psi(q)=1\}=\varnothing$.
\\$1-\varnothing = 1,\quad \varnothing\in1$ 
\\$\Rightarrow$ $\varnothing \in ((1-
\{w\in1: \forall q\in\Un. \psi(q)=1\}
)\cup \psi(x))$
\\\myqed
\end{theorem}
\section{Translation of all symbols of Isabelle using 3rd approach}
\begin{enumerate}
\item 
$==>::\prop=>\prop=>\prop$
\\$I(==>) \eqdef (x\in 2\mapsto(y\in 2\mapsto \{w\in 1: x=1 \longrightarrow y=1\}))$
\\$I(==>) \in (2\to(2\to2))$

%$I_\statement(==>::\prop=>\prop=>\prop)\eqdef$\\$=(x\in 2\mapsto(y\in 2\mapsto \{w\in 1: x=1 \longrightarrow y=1\}))\in (2\to(2\to2))$
\item $I_\statement(!!::('a=>\prop)=>\prop)=$\\$=(g\in(v('a)\to2)\mapsto \{w\in1: \forall q\in2.g(q)=1\})\in ((v('a)\to2)\to2)$ %ok
\item $I_\statement(\Truej::o=>\prop)=$
\\$=\id_2\in(2\to2)=(x\in2 \mapsto \{w\in1: x=1\})\in(2\to2) $
%\\$=I_{\term}(\Truej)\in I_\type(o=>\prop)$
%\\$=1_{\{\varphi\in\Fm : \Tau \vdash \varphi\}} \in(\Fm\to 2)$
\item $I_\statement(\False::o)=0\in2$
\item $I_\statement(\longrightarrow::o=>o=>o)=$
\\$=(x\in 2\mapsto(y\in 2\mapsto 
(1-x)\cup y))\in (2\to(2\to2))=$
\\$=(x\in 2\mapsto(y\in 2\mapsto \{m\in1: x=0 \lor y=1\}))$
%\\$=(\psi_1\in\Fm\mapsto(\psi_2\in\Fm\mapsto ``("+\psi_1+``\longrightarrow"+\psi_2+``)"))\in(\Fm\to(\Fm\to\Fm))$
\item $I_\statement(\forall::((i=>o)=>o))=$
%\\$=(g\in(\Term\to\Fm)\mapsto ``(\forall"+?+``."+*+``)")\in((\Term\to\Fm)\to\Fm)$
\\$=(g\in(\Un\to2)\mapsto \{w\in1: \forall q\in\Un.g(q)=1\})\in ((\Un\to2)\to2)$
\item ...

%(\varphi \mapsto \Tau \vdash \varphi) $
%\item $I_{\term}(A=>B)=I_{\type}(A) $
%\item $I_{\term}(i)=\Term$
%\item $I_{\term}(o)=\Fm$
%\item $I_{\term}(A\Rightarrow B)=I_{type}(A)\rightarrow I_{type}(B)$ (set of functions)
\end{enumerate}


\ \\...\\
...\\
...\\
\section*{References}
*  Barwise, Jon (ed.) (1977). \textit{Handbook of Mathematical Logic}. North-Holland.
%\myprf
%\myqed
\end{document}