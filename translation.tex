\documentclass[10pt,a4paper]{article}
\usepackage[utf8]{inputenc}
\usepackage{amsmath}
\usepackage{amsfonts}
\usepackage{amssymb}
\usepackage{float}
\usepackage{hyperref}
\newcommand{\Tau}{\mathcal{T}}  % \mathrm{\mathcal{T}}}

\usepackage{amsthm} % theorems
\usepackage{mathtools} % \coloneqq  ( tlmgr install mathtools )
% \usepackage[lite]{mtpro2}
\theoremstyle{definition}
\newtheorem{definition}{Definition}[section]
\newtheorem{lemma}{Lemma}[section]
\newtheorem{theorem}{Theorem}[section]

\newtheorem{exercise}{Exercise}[section]
\newcommand{\Fun}{{\mbox{Fun}}}
\newcommand{\Rel}{{\mbox{Rel}}}
\newcommand{\dom}{{\mbox{dom}}}
\newcommand{\apo}{{\mbox{'}}}  % apostrophe
\newcommand{\The}{{\mbox{The\,}}}
\newcommand{\Ob}{{\mbox{Ob}}}
\newcommand{\Mor}{{\mbox{Mor}}}
\newcommand{\Cat}{{\mbox{Cat}}}
\newcommand{\Hom}{{\mbox{Hom}}}
\newcommand{\Nat}{{\mbox{Nat}}}
\newcommand{\HomMor}{{\widetilde{\Hom}\mbox{}}}
\newcommand{\FMor}{{\widetilde{F}\mbox{}}}
\newcommand{\GMor}{{\widetilde{G}\mbox{}}}
\newcommand{\op}{{\mbox{op}}}
\newcommand{\id}{{\mbox{id}}}
\newcommand{\mor}{{\mbox{mor}}}
\newcommand{\LRA}{\Longleftrightarrow}
\newcommand{\defi}{{\mbox{def}}}
\newcommand{\eqdef}{{\stackrel{\defi}{=}}}
\newcommand{\propdef}{{\stackrel{\defi}{\ \Longleftrightarrow\ }}}
\newcommand{\inter}{{\bigcap}}
\newcommand{\interclass}{{{\bigcap}_C}}
\newcommand{\Set}{{\mbox{Set}}}
\newcommand{\myprf}{\noindent\textbf{Proof.}}
% quod erat demonstrandum :
\newcommand{\myqed}{\noindent\textbf{Q.E.D.}} 

\begin{document}
Our aim is to translate Isabelle's term into theorem statement, which is about derivability of certain formulas in the predicate calculus.

The following assignment of the proof constant to some term  
$$\mbox{exE}:``[| \exists x.P(x); \bigwedge x. P(x) \Longrightarrow R|] \Longrightarrow R"$$
is a notation for
$$\mbox{exE}:``\exists x.P(x) \Longrightarrow \left(\left(\bigwedge x. P(x) \Longrightarrow R\right) \Longrightarrow R\right)".$$

The following assignment
$$\mbox{impI}:``(A\Longrightarrow B)\Longrightarrow (A\longrightarrow B)"$$
gives us a hint that impI should be treated as the deduction theorem.

\begin{definition}
 $\Tau$ that is a complete theory iff
$$ \forall \varphi. (\Tau \vdash \phi) \lor (\Tau \vdash \neg\varphi).$$
\end{definition}
% \Tau \nvdash \phi

\begin{theorem}
Let $\Tau$ be a complete first-order theory and $\alpha,\beta$ are formulas in a language of that theory, then
$$ ((\Tau \vdash \alpha)\Longrightarrow (\Tau \vdash \beta)) \Longrightarrow (\Tau \vdash (\alpha\longrightarrow \beta)),$$
where $\Longrightarrow$ is the symbol for implication of metatheory.\\
%\proof\ \\
\myprf\\
Assume H:$((\Tau \vdash \alpha)\Longrightarrow (\Tau \vdash \beta))$.\\
To prove $\Tau \vdash \alpha \longrightarrow \beta$ it is enough to show that $\Tau, \alpha \vdash \beta$ and then apply deduction theorem. So now our aim is $\Tau, \alpha \vdash \beta$.\\
The theorey $\Tau$ is complete, therefore there are two possibilities:\\
1) $\Tau \vdash \alpha$.\\
Therefore, $\Tau \vdash \beta$ by the hypothesis H and Modus Ponens.\\ Then, by weakening, $\Tau, \alpha \vdash \beta$\\
2) $\Tau \vdash \neg\alpha$.\\
Then $\Tau, \alpha \vdash \neg\alpha$ by weakening.\\ $\Tau, \alpha \vdash \bot$ by axiom for negation and MP.\\ So $\Tau, \alpha \vdash \beta$ by \textit{``ex falso quodlibet''} principle.\\
\myqed
%\qed
\end{theorem}
Let's 

\begin{definition}
 $I=J_{[]}()$ is a translation from Isabelle/ZF's term to metatheory for first-order logic. We choose this metatheory to be also a set theory. $$I : \mbox{IsabelleZFTerms} \to \mbox{FOLFormulas}.$$
 $J$ has an argument $l$ which is some ordered list of zeroes and ones. (It reperesents binary form of the numbers of variables. It will be used later to obtain a variable that is guranteed to be new in the formula.)\\
 
$J_l(t)$ is defined inductively:\\
$J_l(\varphi) = ``(\Tau \vdash \varphi)"$ where $\phi$ is any term of type $o$.\\
$J_l(X ==> Y) = (J_{l+[0]}(X) \longrightarrow J_{l+[1]}(Y))$ where $X, Y$ are terms of type $prop$ and ``+" is a lists' concatenation.\\
Now the complicated part:\\
We need to  translate
$J_l(\bigwedge S. P)$, where $S$ is a symbol of some type and $P$ is some term of type $prop$. (It may contain $S$ variable.)\\
First order theories has only predicate and functional symbols in their signature.\\
$$\mbox{Sign}(\Tau) =\mbox{Const}\cup \mbox{Func} \cup \mbox{Pred}.$$
We assume that our object $\Tau$ theory is already extended by all possible definitions of terms and formulas. (\url{https://en.wikipedia.org/wiki/Extension_by_definitions})

\end{definition}

%\myprf
%\myqed
\end{document}